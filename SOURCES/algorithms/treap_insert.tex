% treap_insert.tex
% 11-21-2001
\nopagenumbers
\parskip=2pt
\parindent=0pt

{\sl An Iterative Algorithm for Treap Insertion
	\hfill \rm Farooq Mela $<$farooq.mela@gmail.com$>$}

\medskip

\noindent {\bf Algorithm I} ({\it Treap Insertion}).
Given a set of nodes which form a treap {\tt T}, and a key to insert
{\tt K}, this algorithm will insert the node into the treap while maintaining
it's heap properties. Each node is assumed to contain {\tt KEY}, {\tt PRIO},
{\tt LLINK}, {\tt RLINK}, and {\tt PARENT} fields. For any given node {\tt N},
{\tt KEY(N)} gives the key field of {\tt N}, {\tt PRIO(N)} gives the priority
field of {\tt N}, {\tt LLINK(N)} and {\tt RLINK(N)} are pointers to {\tt N}'s
left and right subtrees, respectively, and {\tt PARENT(N)} is a pointer to the
node of which {\tt N} is a subtree. Any or all of these three link fields may
be $\Lambda$, which for {\tt LLINK(N)} and {\tt RLINK(N)} indicates that {\tt
N} has no left or right subtree, respectively, and for {\tt PARENT(N)}
indicates that {\tt N} is the root of the treap. The treap has a field
{\tt ROOT} which is a pointer to the root node of the treap.

\medskip
You can find an implementation of this algorithm, as well as many others, in
{\bf libdict}, which is available on the web at
{\tt http://github.com/fmela/libdict}.
\medskip

\parindent=36pt

\item{\bf I1.} [Initialize.]
Set ${\tt N} \gets {\tt ROOT(T)}$, ${\tt P} \gets \Lambda$.

\item{\bf I2.} [Find insertion point.]
If ${\tt N} = \Lambda$, go to step I3.
If ${\tt K} = {\tt KEY(N)}$,
the key is already in the treap and the algorithm terminates with an error.
Set ${\tt P} \gets {\tt N}$;
if ${\tt K} < {\tt KEY(N)}$,
then set ${\tt N} \gets {\tt LLINK(N)}$,
otherwise set ${\tt N} \gets {\tt RLINK(N)}$. Repeat this step.

\item{\bf I3.} [Insert.]
Set ${\tt N} \gets {\tt AVAIL}$.
If ${\tt N} = \Lambda$, the algorithm terminates with an out of memory error.
Set ${\tt KEY(N)} \gets {\tt K}$,
${\tt LLINK(N)} \gets {\tt RLINK(N)} \gets \Lambda$, and
${\tt PARENT(N)} \gets {\tt P}$.
Set {\tt PRIO(N)} equal to a random integer.
If ${\tt P} = \Lambda$, set ${\tt ROOT(T)} \gets {\tt N}$, and go to step I5.
If ${\tt K} < {\tt KEY(P)}$,
set ${\tt LLINK(P)} \gets {\tt N}$; otherwise,
set ${\tt RLINK(P)} \gets {\tt N}$.

\item{\bf I4.} [Sift up.]
If ${\tt P} = \Lambda$ or ${\tt PRIO(P)} \leq {\tt PRIO(N)}$, go to step I5.
If ${\tt LLINK(P)} = {\tt N}$, rotate {\tt P} right; otherwise,
rotate {\tt P} left. Then set ${\tt P} \gets {\tt PARENT(N)}$, and repeat this
step.

\item{\bf I5.} [All done.] The algorithm terminates successfully.

\medskip
\parindent=0pt
{\bf Rotations}

\noindent {\bf Algorithm R} ({\it Right Rotation}).
Given a treap {\tt T} and a node in the treap {\tt N}, this routine will rotate
{\tt N} right.

\parindent=36pt
\item{\bf R1.} [Do the rotation.]
Set
${\tt L} \gets {\tt LLINK(N)}$ and
${\tt LLINK(N)} \gets {\tt RLINK(L)}$.
If ${\tt RLINK(L)} \neq \Lambda$,
then set ${\tt PARENT(RLINK(L))} \gets {\tt N}$.
Set ${\tt P} \gets {\tt PARENT(N)}$, ${\tt PARENT(L)} \gets {\tt P}$.
If ${\tt P} = \Lambda$, then set ${\tt ROOT(T)} \gets {\tt L}$;
if ${\tt P} \neq \Lambda$ and ${\tt LLINK(P)} = {\tt N}$,
set ${\tt LLINK(P)} \gets {\tt L}$, otherwise
set ${\tt RLINK(P)} \gets {\tt L}$.
Finally, set ${\tt RLINK(L)} \gets {\tt N}$,
and ${\tt PARENT(N)} \gets {\tt L}$.

\parindent=0pt

\medskip
The code for a left rotation is symmetric. At the risk of being repetitive, it
appears below.
\medskip

\noindent {\bf Algorithm L} ({\it Left Rotation}).
Given a treap {\tt T} and a node in the treap {\tt N}, this routine will rotate
{\tt N} left.

\parindent=36pt
\item{\bf L1.} [Do the rotation.]
Set
${\tt R} \gets {\tt RLINK(N)}$ and
${\tt RLINK(N)} \gets {\tt LLINK(R)}$.
If ${\tt LLINK(R)} \neq \Lambda$,
then set ${\tt PARENT(LLINK(R))} \gets {\tt N}$.
Set ${\tt P} \gets {\tt PARENT(N)}$, ${\tt PARENT(R)} \gets {\tt P}$.
If ${\tt P} = \Lambda$, then set ${\tt ROOT(T)} \gets {\tt R}$;
if ${\tt P} \neq \Lambda$ and ${\tt LLINK(P)} = {\tt N}$,
set ${\tt LLINK(P)} \gets {\tt R}$, otherwise
set ${\tt RLINK(P)} \gets {\tt R}$.
Finally, set ${\tt LLINK(R)} \gets {\tt N}$,
and ${\tt PARENT(N)} \gets {\tt R}$.

\parindent=0pt

\bye
% EOF
